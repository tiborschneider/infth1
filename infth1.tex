%%This is a very basic article template.
%%There is just one section and two subsections.
\documentclass[a4paper]{article}

%definitions for Formelsammlung

\usepackage[left=1.5cm,right=1.5cm,top=2.5cm,bottom=2cm,landscape]{geometry} 
\usepackage{multicol}
\usepackage[ngerman]{babel}
\usepackage{tabularx}
\usepackage{mathpazo}
\usepackage{mathtools}
\usepackage{amsmath}  
\usepackage{setspace} 
\usepackage{commath}
\usepackage[utf8]{inputenc}
%\usepackage[ansinew]{inputenc}  
\usepackage[T1]{fontenc}
\usepackage{lmodern} 
\usepackage{hyperref}
\usepackage{bigints}
\usepackage{array}
\usepackage[table]{xcolor}
\usepackage{layouts}
\usepackage{siunitx}
\usepackage{wrapfig}
\usepackage{multirow,bigstrut}
\usepackage{trfsigns}
\usepackage{amssymb} 
\usepackage{fancyhdr}
\usepackage{datetime}
\usepackage{pgfplots}
\usepgfplotslibrary{fillbetween}
\usepackage{listings}
\usepackage{mathrsfs}
\usepackage{tabu}
\usepackage{pdflscape}
\usepackage{booktabs}
%\usepackage{mathabx}
\usepackage{graphicx}
\usepackage{supertabular}
\usepackage{siunitx}
\usepackage[europeanvoltages, europeancurrents, europeanresistors, americaninductors, smartlabels]{circuitikz}
\usepackage{tikz-3dplot}

\DeclareMathOperator\arctanh{arctanh}
\DeclareMathOperator\arsinh{arsinh} 
\DeclareMathOperator\arcosh{arcosh}
\DeclareMathOperator\artanh{artanh}
\DeclareMathOperator\arcoth{arcoth} 
\DeclareMathOperator\sinc{sinc} 
\DeclareMathOperator\sgn{sgn} 
\DeclareMathOperator\LPF{LPF} 
\DeclareMathOperator\Q{Q} 
\DeclareMathOperator\erf{erf} 
\DeclareMathOperator\var{Var} 
\DeclareMathOperator\Cov{Cov} 
\DeclareMathOperator\floor{floor} 
\DeclareMathOperator\E{E} 
\DeclareMathOperator\Pr{Pr} 
\DeclareMathOperator\NDFT{DFT} 
\DeclareMathOperator\IDFT{IDFT} 
\DeclareMathOperator\grad{grad} 
\DeclareMathOperator{\divergence}{div}
\DeclareMathOperator{\curl}{curl}


%colorCodes
\definecolor{listinggray}{gray}{0.9}
\definecolor{lbcolor}{rgb}{0.95,0.95,0.95}
\definecolor{lightGray}{gray}{0.1}

\definecolor{cOrange}{HTML}{996633}
\definecolor{clOrange}{HTML}{DBB48D}
\definecolor{cBlue}{HTML}{336699}
\definecolor{clBlue}{HTML}{A0BCD8}
\definecolor{cGreen}{HTML}{339966}
\definecolor{clGreen}{HTML}{94D4B4}
\definecolor{cRed}{HTML}{993333}
\definecolor{clRed}{HTML}{D0B0B0}
\definecolor{cGray}{gray}{0.4}
\definecolor{clGray}{gray}{0.96}



\setlength{\parindent}{0pt}
%\DeclareMathOperator\arctanh{arccot}
\newcolumntype{L}[1]{>{\raggedright\let\newline\\\arraybackslash\hspace{0pt}}m{#1}}
\newcolumntype{C}[1]{>{\centering\let\newline\\\arraybackslash\hspace{0pt}}m{#1}}
\newcolumntype{R}[1]{>{\raggedleft\let\newline\\\arraybackslash\hspace{0pt}}m{#1}}
\newcolumntype{Y}{>{\centering\arraybackslash}X}
\newcolumntype{Z}{>{\raggedleft\arraybackslash}X}
\newcommand{\fmm}{\displaystyle} 
\newcommand{\cn}[1]{\underline{#1}} 
\newcommand{\hlaplace}{\quad\laplace\quad}
\newcommand{\hLaplace}{\quad\Laplace\quad}
\newcommand{\ztransform}{\, \, \xrightarrow{\, \, z\, \, } \, \,}
\newcommand{\zTransform}{\, \, \xrightarrow{\, \, z^{-1}\, \, } \, \, }
\newcommand{\infint}{\int_{-\infty}^{+\infty}}
\newcommand{\infiint}{\iint_{-\infty}^{+\infty}}
\newcommand{\limint}{\lim_{T\rightarrow \infty} \frac{1}{T} \int_{-T/2}^{T/2}}
\newcommand{\bedeq}{\mathrel{\stackrel{\makebox[0pt]{\mbox{\normalfont\tiny WSS}}}{=\joinrel=}}}
\renewcommand{\fourier}{\mathcal{F}}
\newcommand{\infsum}[1]{\sum_{#1 = -\infty}^{\infty} }
\newcommand{\cif}{\text{if}\:}
\newcommand{\cand}{\:\text{and}\:}
\newcommand{\celse}{\text{otherwise}\:}
\renewcommand{\abs}[1]{\left| #1 \right|}
\newcommand{\cvec}[1]{\left[\begin{smallmatrix} #1 \end{smallmatrix}\right]}
\newcommand{\vvec}[1]{\renewcommand*{\arraystretch}{0.8}\left[\begin{array}{c} #1 \end{array}\right]}
\renewcommand{\hat}[1]{\widehat{#1}}
\let\oldsi\si
\renewcommand{\si}[1]{\; \left[\oldsi[per-mode = fraction]{#1}\right]}
\newcommand{\ex}[1]{\textcolor{cBlue}{#1}}
\newcommand{\pdif}[2]{\frac{\partial #1}{\partial #2}}
\newcommand{\pdiff}[2]{\frac{\partial^2 #1}{\partial #2 ^2}}

\newenvironment{cmatrix}
    {\left( \begin{matrix}}
    {\end{matrix} \right)}

\newcommand{\plotTF}[1]{
\begin{tikzpicture}
\begin{axis}[xlabel=$\omega$,ylabel=$\abs{H(\omega}$, xmin = 0, xmax = 3.5, ymin = 0, ymax = 1, xtick = {3.14}, xticklabel={$\pi$}, ytick={0}, axis lines=middle, width=6cm, height=4cm, 
every axis x label/.style={at={(ticklabel* cs:1.05)},anchor=west}]
\addplot[name path=H, domain=0:3.14, samples=200] {#1};
\path[name path=axis] (axis cs:0,0.01) -- (axis cs:3.14,0.01);
\addplot[fill=clGray] fill between[of=H and axis, soft clip={domain=0.01:3.14}];
\end{axis}
\end{tikzpicture}
}

\newenvironment{donotbrake}{\begin{minipage}{\columnwidth}}{\end{minipage} \vspace{1em}}

\newenvironment{cmat}[1]{
  \renewcommand*{\arraystretch}{0.9}
  \left[
  \begin{array}{#1}
}{
  \end{array}
  \right]
}

\newenvironment{case}{
  \left\{ \begin{array}{ll}
}{
  \end{array} \right.
}


\newenvironment{scase}{
  \renewcommand*{\arraystretch}{1}
  \left\{ \begin{array}{ll}
  }{
  \end{array} \right.
}

\newcommand\xdownarrow[1][2ex]{%
   \mathrel{\rotatebox{90}{$\xleftarrow{\rule{#1}{0pt}}$}}
}

\newcommand{\ncr}[2]{\binom{#1}{#2}}

\renewenvironment{description}{\color{cGray}}{}
\newenvironment{definition}{\color{cGray}}{} 
\newcommand{\cdef}[1]{\begin{definition}#1\end{definition}}


\newcommand{\vLaplace}[1][]{\mbox{\setlength{\unitlength}{0.1em}%
        \begin{picture}(10,20)%
          \put(3,2){\circle{4}}%
          \put(3,4){\line(0,1){12}}%
          \put(3,18){\circle*{4}}%
          \put(10,7){#1}
        \end{picture}%
       }%
 }%

\newcommand{\vlaplace}[1][]{\mbox{\setlength{\unitlength}{0.1em}%
        \begin{picture}(10,20)%
          \put(3,2){\circle*{4}}%
          \put(3,4){\line(0,1){12}}%
          \put(3,18){\circle{4}}%
          \put(10,7){#1}
        \end{picture}%
       }%
 }%         
 
           
\newenvironment{blockdiagram}[1]{
	\begin{tikzpicture}
		[auto, node distance=2.5cm,>=latex', scale=#1, every node/.style={scale=#1}]
}{
	\end{tikzpicture}
} 
 
 
\renewcommand{\arraystretch}{1.5}


\newenvironment{mtabular}[1] {
  \renewcommand{\arraystretch}{2}
  
  \begin{tabular}{#1}
}  
{
  \end{tabular}
  
  \renewcommand{\arraystretch}{1.5}
}

\newenvironment{lmtabular}[1] {
\renewcommand{\arraystretch}{2}

\begin{supertabular}{#1}
}  
{
\end{supertabular}

\renewcommand{\arraystretch}{1.5}
}

\newenvironment{dtabular} {
  \begin{tabular}{>{\begin{definition}}l<{\end{definition}} >{\begin{definition}}l<{\end{definition}}}
}  
{
  \end{tabular}
}

\newenvironment{ddtabular} {
  \begin{center}
  \begin{tabular}{>{\begin{definition}}l<{\end{definition}} >{\begin{definition}}l<{\end{definition}} | >{\begin{definition}}l<{\end{definition}} >{\begin{definition}}l<{\end{definition}}}
}  
{
  \end{tabular}
  \end{center}
}


%configure tikz
%system description
\usetikzlibrary{shapes,arrows}
\usetikzlibrary{decorations.markings}
\tikzstyle{block} = [draw, rectangle, minimum height=2.5em, minimum width=5em]
\tikzstyle{input} = [coordinate]
\tikzstyle{output} = [coordinate]
\tikzstyle{pinstyle} = [pin edge={to-,thin,black}]
\tikzstyle{sum} = [draw, circle, node distance=1em, minimum height=1.5em]
\tikzset{
	>=latex,
	photon/.style={decorate,decoration={snake,post length=1mm}}
}
\tikzset{->-/.style={decoration={
			markings,
			mark=at position #1 with {\arrow{>}}},postaction={decorate}}}
\tikzset{%
  block/.style    = {draw, thick, rectangle, minimum height = 2.5em,
    minimum width = 3em},
  sum/.style      = {draw, circle, node distance = 1.5cm}, % Adder
  input/.style    = {coordinate}, % Input
  output/.style   = {coordinate}, % Output
  gain/.style     = {draw, thick, isosceles triangle, minimum height = 2em, isosceles triangle apex angle=60},
  rgain/.style    = {draw, thick, isosceles triangle, minimum height = 2em, isosceles triangle apex angle=60}
}


%externalize TIKZ
%\usetikzlibrary{external}
%\tikzexternalize[prefix=tikz/]

%lstlisting

\lstset{
  backgroundcolor=\color{lbcolor},
  tabsize=2,    
% rulecolor=,
  language=[GNU]C++,
  basicstyle=\scriptsize,
  upquote=true,
  aboveskip={1.5\baselineskip},
  columns=fixed,
  showstringspaces=false,
  extendedchars=false,
  breaklines=true,
  prebreak = \raisebox{0ex}[0ex][0ex]{\ensuremath{\hookleftarrow}},
  frame=single,
  numbers=none,
  showtabs=false,
  showspaces=false,
  showstringspaces=false,
  identifierstyle=\ttfamily,
  keywordstyle=\color{cBlue}
  commentstyle=\color{cGreen},
  stringstyle=\color{cRed},
  numberstyle=\color{black},
% \lstdefinestyle{C++}{language=C++,style=numbers}’.
}
\lstset{
  backgroundcolor=\color{lbcolor},
  tabsize=2,
  language=C++,
  captionpos=b,
  tabsize=3,
  frame=lines,
  numbers=none,
  numberstyle=\tiny,
  numbersep=5pt,
  breaklines=true,
  showstringspaces=false,
  basicstyle=\ttfamily,
  identifierstyle=\color{cOrange},
  keywordstyle=\color{cBlue},
  commentstyle=\color{cGreen},
  stringstyle=\color{cRed}
}

\lstdefinelanguage{makefile}{
  morekeywords={cc,CFLAGS,LFLAGS,OBJ,EXE},
  morecomment=[l]{\#}
}

\lstdefinestyle{makefile}{
  language=makefile,
  basicstyle=\ttfamily,
  keywordstyle=\color{cBlue},
  commentstyle=\color{cGreen},
  frame=lines,
  numbers=none,
  backgroundcolor=\color{lbcolor}
}


\newcolumntype{M}{>{$}c<{$}} % math-mode version of "c" column type

%header & footer
\pagestyle{fancy}
\lhead{Tibor Schneider}
\rhead{Seite \thepage}
\cfoot{\today} 

\renewcommand{\headrulewidth}{0.4pt}
\renewcommand{\footrulewidth}{0.4pt}

\newtheorem{theorem}{Theorem}

%Title of Document
\chead{Information Theory I \@ Summary} 

\begin{document}
\begin{twocolumn} 

\section{Definitions}

\begin{itemize}
    \item \textbf{Chance Variable}: Outcome of random Experiment, where the outcome does not need to be a number
    \item \textbf{Random Variable}: ($X \in \mathbb{R}$) is random wit a certan probability distribution. From a random variable, one can compute $X^2$, $\E[X], \var(X)$.
    \item \textbf{Probability Mass Funtion (PMF)}: $\Pr[X=x] = \mathcal{P}(x) = \mathcal{P}_X(x)$
    \item \textbf{Alphabet}: ($X \in \mathcal{X}$), where $\mathcal{X}$ is the alphabet
\end{itemize}

\subsection{Uncertanty (Entropy)}

Uncertanty of a random variable is defined as:
\[H(X) = \sum_{x \in \mathcal{X}} \mathcal{P}(x) \log \frac{1}{\mathcal{P}(x)} = E_X \left[ \log \frac{1}{\mathcal{P}(x)} \right] \qquad 0 \leq H(X) \leq \log \abs{\mathcal{X}}\]

\begin{itemize}
    \item Lower bound: $H(X) = 0$ if and only if $\exists x \in \mathcal{X}$ s.t. $\mathcal{P}(x) = 1$
    \item Upper bound: $H(X) = \log \abs{\mathcal{X}}$ if and only if $\mathcal{P}(x) = \frac{1}{\abs{\mathcal{X}}} \forall x \in \mathcal{X}$
\end{itemize}

Note that the basis of the logarithm defines the unit. For base 2: the unit is called a \textbf{bit}, and for the natural logarithm, the unit is called \textbf{nat} (for natural).

$H_b$ denotes binary uncertanty. Let $X \in \{0,1\}$ and $\matcal{P}(X=1) = 1-\mathcal{P}(X=0) = p$. Then:
\[H_b = p \log \frac{1}{p} + (1-p) \log \frac{1}{1-p}\]

\subsubsection{Relative Entropy}

We want to compare two chance variables the same alphabet $\mathcal{X}$ with different PMF $p(x)$ and $q(x)$. The relative entropy is defined as:
\[D\left(p(\cdot) || q(\cdot) \right) \triangleq \sum_{x \in \mathcal{X}} p(x) \log \frac{p(x)}{q(x)}\]
\[D(p||q) \geq 0 \quad \text{and} \quad D(p||q) = 0 \text{ if and only if } p = q\]
We have the inequality, that $H(X,Y) \leq H(X) + H(Y)$ with equality $H(X,Y) = H(X) + H(Y)$ if and only if $X$ and $Y$ are independent.

\subsubsection{Joint Uncertanty}
We have two chance variables $X \in \mathcal{X}$ and $Y \in \mathcal{Y}$ with PMF $\mathcal{P}(x,y)$. The Joint Uncertanty is:
\[H(X,Y) = \sum_{x \in \mathcal{X},\ y \in \mathcal{Y}} \mathcal{P}(x,y) \log \frac{1}{\mathcal{P}(x,y)} = E_{X,Y} \left[ -\log \mathcal{P}(x,y) \right]\]

The \textbf{chain rule} for Joint Entropy is:
\[H(X,Y) = H(X) + H(Y|X) = H(Y) + H(X|Y)\]
\[H(X_1,X_2,\ldots,X_n) = H(X_1) + H(X_2|X_1) + \cdots + H(X_n|X_1,\ldots,X_{n-1})\]

We introduce the \textbf{Notation}: $X_k^n \mapsto X_k,\ldots,X_n$ and $X^n \mapsto X_1^n$. Then, the chain rule can be written as:
\[H(X_1^n) = H(X^n) = \sum_i^n H(X_i|X_1^{i-1}) = \sum_i^n H(X_i|X_{i+1}^n)\]


\subsubsection{Conditional Entropy}
The conditional entropy given the Event $Y = y \in \mathcal{Y}$ 
\[H(X|Y=y) = \sum_{x \in \mathcal{X}} \mathcal{P}(x|Y=y) \log \frac{1}{\mathcal{P}(x|Y=y)}\]
\[H(X|Y) = \sum_{y \in \mathcal{Y}} \mathcal{P}(y) H(X|Y=y) = \sum_{x \in \mathcal{X},\ y \in \mathcal{Y}} \mathcal{P}(x,y) \log \frac{1}{\mathcal{P}(x|y)}\]
With the information from $Y$, we know more about $X$: $H(X) \geq H(X|Y)$

\subsubsection{Fano's Inequality}
Consider $X \in \mathcal{X}$ and $Y \in \mathcal{Y}$ with a joint PMF $\mathcal{P}(X,Y)$. 
We want to guess $X$ from $Y$ with a function $g(Y): \mathcal{Y} \mapsto \mathcal{X}$
We define $\mathcal{P}_e = \mathcal{P}\left[ g(Y) \neq X \right]$.
No matter how $g(\cdot)$ is defined, the following inequality holds:
\[ 1 + \mathcal{P}_e \log \abs{\mathcal{X}} \geq H_b(\mathcal{P}_e) + \mathcal{P}_e \log \left( \abs{\mathcal{X}-1} \right) \geq H(X|Y) \]

\subsection{Mutual Information}
Mutual Information is the amount of entropy that goes down as soon as you know $Y$:
\[I(X;Y) = H(X) - H(X|Y) = H(Y) - H(Y|X)\]

\begin{itemize}
    \item $I(X;Y) \geq 0$ with equality $I(X;Y) = 0$ if and only if $X$ and $Y$ are independent
    \item $I(X;Y) = I(Y;X)$
    \item $I(X;Y) = H(X) + H(Y) - H(X|Y)$
    \item $I(X;Y) = D(\mathcal{P}_{XY}||\mathcal{P}_X \cdot \mathcal{P}_Y)$
    \item $I(X;Y|Z) = \sum_{z \in \mathcal{Z}} \mathcal{P}_Z(z) \cdot I(X;Y|Z=z)$
\end{itemize}

\section{Data Crompression}

\subsection{Definitions}

\begin{itemize}
    \item The \textbf{code} $\mathcal{C}(X)$ is a Mapping $\mathcal{C}:\ \mathcal{X} \mapsto {\{0,1\}}^+$.
        The Output ${\{0,1\}}^+$ is a string of finite nonzero length.
    \item A \textbf{one-to-variable} code $\mathcal{C}$ which maps one random variable to a striing of variable length. 
        In comparison, a \textbf{one-to-fixed} code $\mathcal{C}$ maps the input to a string of fixed length.
    \item The code $\mathcal{C}$ is \textbf{nonsingular} if $\mathcal{C}(x) = \mathcal{C}(x') \mapsto x = x'$
    \item The \textbf{extension} $\mathcal{C}^+$ of a code $\mathcal{C}$ is a mapping $\mathcal{C}^+: \mathcal{X}^+ \mapsto {\{ 0,1 \}}^+$ and is defined as:
        \[ \mathcal{C}^+(x_1, \ldots, x_n) = \mathcal{C}(x_1) \mathcal{C}(x_2) \ldots \mathcal{C}(x_n) \]
    \item The code $\mathcal{C}$ is \textbf{uniquely decodable} (u.d.) if $\mathcal{C}^+$ is nonsingular. 
        (Meaning there cannot be two input strings which produce the same code $\mathcal{C}^+$)
    \item The code $\mathcal{C}$ is \textbf{prefix free} if no codeword is a prefix of another. 
        Every prefix free code is also uniquely decodable.
    \item The \textbf{expected description length} $\mathcal{L}$ is defined as follows, where $\ell(\mathcal{C}(x))$ is the length of the encoded string, and $\mathcal{L}^\ast$ is smallest length possible.
        \[ \mathcal{L} = \sum_{x \in \mathcal{X}} \mathcal{P}_X(x) \cdot \ell(\mathcal{C}(x))\qquad \mathcal{L}^\ast = \min_{\mathcal{C}} \mathcal{L}(\mathcal{C}) \] %TODO
\end{itemize}

\subsection{Kraft's Inequality}

Let $x \in \mathcal{X}$ and let $\ell_1, \ldots, \ell_{\abs{\mathcal{X}}} > 0$ be positive integers. 
Then, there exists a prefix-free (and therefore a uniquely decodable) Code $\mathcal{C}$ with lengths $\ell_1, \ldots, \ell_\abs{\mathcal{X}}$, if those lengths satisfy the following condition:

\[ \sum_{k = 1}^{\abs{\mathcal{X}}} 2^{-\ell_k} \leq 1 \]




\section{General Formulas}
\subsection{Inequalities}

\begin{mtabular}{lll}
    \textbf{Name} & \textbf{condition} &\textbf{Inequality} \\ \toprule
    Uncertanty & $X \in \mathcal{X}$ & $\fmm 0 \leq H(X) \leq \log \abs{\mathcal{X}}$ \\
               & \multicolumn{2}{l}{$H(X) = 0 \iff \exists x \in \mathcal{X}$, s.t. $\mathcal{P}(x) = 1$} \\
               & \multicolumn{2}{l}{$\fmm H(X) = \log \abs{\mathcal{X}} \iff \mathcal{P}(x) = \log \frac{1}{\abs{\mathcal{X}}} \forall x \in \mathcal{X}$} \\
    Rel. Entropy & $X \in \mathcal{X}, \ X \in \mathcal{Y}$ & $H(X,Y) \leq H(X) + H(Y)$ \\
                 & \multicolumn{2}{l}{$H(X,Y) = H(X) + H(Y) \iff X \text{ and } Y \text{ are independent.}$}\\
    Cond. Entropy & $X \in \mathcal{X}$ & $H(X) \geq H(X|Y)$ \\
    
    IT ineq. & Logarithm & $\fmm \left(1-\frac{1}{\xi}\right) \log_b(e) \leq \log_b(\xi) \leq (\xi-1) \log_b(e)$ \\
    Jensen ineq. & $f$ concave in $[a,b]$ & $\E\left[ f(x) \right] \leq f\left(  \right)$ \\
Kraft's ineq. & $\mathcal{C}$ is u.d. & $\sum_{x \in \mathcal{X}} 2^{-\ell(\mathcal{C}(x))} \leq 1$ 

\end{mtabular}


\end{twocolumn}
\end{document}
